%# -*- coding: utf-8 -*-
% !TeX encoding = UTF-8 Unicode
% !TeX spellcheck = en_US
% !TeX TS-program = xelatex
%~ \XeTeXinputencoding "UTF-8"
% vim:ts=4:sw=4
%
% 以上设定默认使用 XeLaTex 编译,并指定 Unicode 编码,供 TeXShop 自动识别

% 使用 LaTex 写中文的配置模板
%\documentclass[12pt]{article}

%\newcommand{\doctitle}{Peer-to-Peer Communication Across Network Address Translators}
%\newcommand{\doctitle}{穿越NAT的点对点通信}
%\newcommand{\docauthor}{Bryan Ford \& Pyda Srisuresh \& Dan Kegel}
%\newcommand{\dockeywords}{NAT穿越, 点对点, NAT Traversal, Peer-to-Peer}
%\newcommand{\docsubject}{NAT穿越}
%\newcommand\usefontspeczh[1]{#1} % use fontspec for zh_CN
%\newcommand\charsetzhcn[1]{#1} % the charset encoding for zh_CN
%\newcommand\formatzhcn[1]{#1} % the page format for zh_CN

\newcommand\mymainfont{DejaVu Serif}%{Times New Roman} %{DejaVu Serif}
\newcommand\myboldfont{WenQuanYi Micro Hei Mono}%{AR PL UKai CN}%{YaHei Consolas Hybrid}%{黑体}%{標楷體}
\newcommand\myitalicfont{DejaVu Serif}%{Times New Roman} %{Garamond}

\renewcommand\myboldfont{Adobe Garamond Pro Bold}
\renewcommand\mymainfont{Adobe Garamond Pro}
\renewcommand\myitalicfont{Adobe Garamond Pro Italic}


\newcommand\mymonofont{DejaVu Sans Mono}%{FreeMono} %{WenQuanYi Micro Hei Mono} %{Monaco}
\newcommand\mysansfont{DejaVu Sans}%{FreeSans}

\newcommand\mycjkboldfont{WenQuanYi Micro Hei Mono}%{Adobe Heiti Std}%{AR PL UKai CN}%{YaHei Consolas Hybrid}%{黑体}%{標楷體}
\newcommand\mycjkitalicfont{全字庫正楷體} %{Adobe Kaiti Std}
\newcommand\mycjkmainfont{AR PL UMing CN}%{Adobe Song Std}%{仿宋}%{宋体}%{新宋体}%{文鼎PL新宋}%
\newcommand\mycjksansfont{Adobe Ming Std}
\newcommand\mycjkmonofont{WenQuanYi Micro Hei Mono}%{AR PL UMing CN}%{WenQuanYi Micro Hei Mono}

\usepackage{ifthen}
\usepackage{ifpdf}
\usepackage{ifxetex}
\usepackage{ifluatex}

\usepackage{color}
\usepackage[rgb,x11names]{xcolor} %must before tikz, x11names defines RoyalBlue3


\usefontspeczh{

\usepackage[cm-default]{fontspec} % XeLaTex 配合 fontspec

\charsetzhcn{
  %\renewcommand\mycjkmainfont{AR PL UMing CN}%{仿宋}%{宋體}%{新宋體}%{文鼎PL新宋}%

  \ifxetex    % xelatex

    %\usepackage[cm-default]{fontspec} % XeLaTex 配合 fontspec 可以非常方便地设置字体。[cm-default]选项主要用来解决使用数学环境时数学符号不能正常显示的问题
    %\usepackage{xltxtra,xunicode} %这行和上行 \usepackage[cm-default]{fontspec} 解决公式不正常的问题.但是打开后有些如 itemize 的点不能显示。

    \usepackage[
        BoldFont, % 允許粗體
        SlantFont,        % 允許斜體
        %CJKsetspaces,
        CJKchecksingle
        ]{xeCJK}
    \defaultfontfeatures{Mapping=tex-text} %如果沒有它,會有一些 tex 特殊字符無法正常使用,比如連字符。

    \XeTeXlinebreaklocale "zh"                      % 重要,使得中文可以正確斷行!
    \XeTeXlinebreakskip = 0pt plus 1pt minus 0.1pt  %

    \setCJKmainfont[BoldFont=\mycjkboldfont, ItalicFont=\mycjkitalicfont]{\mycjkmainfont}
    \setCJKsansfont{\mycjksansfont}%{Adobe Ming Std} %{AR PL UMing CN} %{Microsoft YaHei}
    \setCJKmonofont{\mycjkmonofont}

\setmonofont[Scale=0.8]{\mymonofont} % 英文等宽字体
\setsansfont{\mysansfont}       % 英文无衬线字体
\setmainfont[Ligatures=TeX]{\mymainfont}        % 英文衬线字体, setmainfont=setromanfont
\setromanfont[Mapping=tex-text,  % 沿用 LaTex 的一些习惯的标点转换,例如 en-dash 以两个减号表示
    Ligatures={TeX,Required,Common}, % 如果此字体内置 Ligatures 定义则启用
    ItalicFont={\myitalicfont},  % 斜体用 Times Italic,严格来说只有拉丁子母有斜体。
    BoldFont={\myboldfont}]      % 粗体用字体
    {\mymainfont}                % 内文使用字体, Linux 下用 "fc-list :lang=zh-cn" 列出支持的中文字体


% 设置中文状态下, 恢复 lstlisting 的 mono 字体
\usepackage[T1]{fontenc}
\DeclareTextCommandDefault{\nobreakspace}{\leavevmode\nobreak\ } % EU1 encoding setup has changed \nobreakspace from being an encoding-independent command to an encoding-dependent command, but without setting up a default definition so it works in all encodings.

  \fi


\formatzhcn{
    %\newfontfamily{\j}{Osaka}       % 设置特殊字符,这里是为日文准备的特殊字体。没有该字体,所以关闭。

    \setlength{\parindent}{2.04em}  %设置首行缩进。只有中文才打开。
    \linespread{1.3}                % 设置行距

    %\definecolor{bisque}{rgb}{.996,.891,.755}
    %\pagecolor{bisque} % 设置背景颜色

    % 设置原文照排环境的字体
    \makeatletter
    \def\verbatim@font{\sffamily\small}
    \makeatother

    % 将默认的英文重定义为中文
    \renewcommand{\contentsname}{\cnt{Content}{目录}{目錄}}
    \renewcommand{\listfigurename}{\cnt{List of Figures}{插图目录}{插圖目錄}}
    \renewcommand{\listtablename}{\cnt{List of Tables}{表格目录}{表格目錄}}
    \renewcommand{\indexname}{\cnt{Index}{索引}{索引}}
    \renewcommand{\tablename}{\cnt{Table}{表}{表}}
    \renewcommand{\figurename}{\cnt{Figure}{图}{圖}}
    \renewcommand{\appendixname}{\cnt{Chapter}{附录}{附錄}}

    % article
    %\renewcommand{\refname}{\cnt{Bibliography}{参考文献}{參考文獻}}
    %\renewcommand{\abstractname}{\cnt{Abstract}{摘要}{摘要}}

    % book
    %\renewcommand{\chaptername}{\cnt{Chapter}{章节}{章節}}
    %\renewcommand{\bibname}{\cnt{Bibliography}{参考}{參考}}

    %\renewcommand{\IEEEkeywordsname}{\cnt{Keywords}{关键词}{關鍵詞}}


    % 设置页眉页脚
    \usepackage[pagestyles,compact]{titlesec} % 定制页眉页脚
    \newpagestyle{main}{%
        \sethead[$\cdot$~\thepage~$\cdot$][][\thesection\quad%
        \sectiontitle]{\thesection\quad\sectiontitle}{}{%
            $\cdot$~\thepage~$\cdot$}
        \setfoot{}{}{}\headrule}
        \pagestyle{main}
        \renewpagestyle{plain}{\sethead{}{}{}\setfoot{}{}{}}
    \pagestyle{plain}

    %% 设置chapter, section与subsection的格式
    \titleformat{\chapter}{\centering\huge}{\textbf{第\thechapter{}章}}{1em}{\textbf}
    \titleformat{\section}{\centering\LARGE}{\textbf{\thesection}}{1em}{\textbf}
    \titleformat{\subsection}{\Large}{\textbf{\thesubsection}}{1em}{\textbf}

    %% For LaN
    \newcommand{\LaN}{L{\scriptsize\hspace{-0.47em}\raisebox{0.23em}{A}}\hspace{-0.1em}N}

    %% 去掉表头中的冒号
    \makeatletter
        \long\def\@makecaption#1#2{%
            \vskip\abovecaptionskip
            \sbox\@tempboxa{#1~~#2}%
            \ifdim \wd\@tempboxa >\hsize
                #1~~#2\par
            \else
                \global \@minipagefalse
                \hb@xt@\hsize{\hfil\box\@tempboxa\hfil}%
            \fi
            \vskip\belowcaptionskip}
    \makeatother
} % formatzhcn

} % \charsetzhcn

\setmonofont[Scale=0.8]{\mymonofont} % 英文等宽字体
\setsansfont{\mysansfont}       % 英文无衬线字体
%\setmainfont{\mymainfont}        % 英文衬线字体, setmainfont=setromanfont
\setromanfont[Mapping=tex-text,  % 沿用 LaTex 的一些习惯的标点转换,例如 en-dash 以两个减号表示
    Ligatures={Required,Common}, % 如果此字体内置 Ligatures 定义则启用
    ItalicFont={\myitalicfont},  % 斜体用 Times Italic,严格来说只有拉丁子母有斜体。
    BoldFont={\myboldfont}]      % 粗体用字体
    {\mymainfont}                % 内文使用字体, Linux 下用 "fc-list :lang=zh-cn" 列出支持的中文字体
} % \usefontspeczh

% 设置中文状态下, 恢复 lstlisting 的 mono 字体
\usepackage[T1]{fontenc}
\DeclareTextCommandDefault{\nobreakspace}{\leavevmode\nobreak\ } % EU1 encoding setup has changed \nobreakspace from being an encoding-independent command to an encoding-dependent command, but without setting up a default definition so it works in all encodings.

\usepackage{dtklogos} % \LaTeXe 等
% XeTeX logo
%\def\XeTeX{\leavevmode
%    \setbox0=\hbox{X\lower.5ex\hbox{\kern-.15em\reflectbox{E}}\kern-.1667em \TeX}%
%    \dp0=0pt\ht0=0pt\box0}

% the algorithm2e package
\makeatletter
\newif\if@restonecol
\makeatother
\let\algorithm\relax
\let\endalgorithm\relax
\usepackage[ruled,vlined]{algorithm2e} %\usepackage[figure,ruled,vlined]{algorithm2e}

\usepackage{url}
\usepackage{array}

\usepackage{courier}
\usepackage{listings} % list the source code
\definecolor{ForestGreen}{rgb}{0.13,0.55,0.13}

\lstset{
    language=C,
    captionpos=b, %t,
    tabsize=3,
    basicstyle=\footnotesize\ttfamily, %basicstyle=\ttfamily\normalsize, %basicstyle=\small\ttfamily, %\normalfont\ttfamily, % \large\ttfamily, % \small\ttfamily, % \footnotesize\ttfamily, % \scriptsize\ttfamily, % Standardschrift,
    numbers=left,               %左侧显示行号 往左靠,还可以为right,或none,即不加行号
    stepnumber=1,               %若设置为2,则显示行号为1,3,5,即stepnumber为公差,默认stepnumber=1
    %numberstyle=\tiny,         %行号字体用小号
    numberstyle={\color[RGB]{0,192,192}\tiny} ,%设置行号的大小,大小有tiny,scriptsize,footnotesize,small,normalsize,large等
    numbersep=8pt,              %设置行号与代码的距离,默认是5pt
    breaklines=true,            %对过长的代码自动换行
    showstringspaces=false,     %不显示代码字符串中间的空格标记
    frame=shadowbox, %=lines,                    %把代码用带有阴影的框圈起来
    stringstyle=\ttfamily,      % 代码字符串的特殊格式
    commentstyle=\color{ForestGreen}, %\color{red!50!green!50!blue!50}, %浅灰色的注释
    rulesepcolor=\color{red!20!green!20!blue!20}, %代码块边框为淡青色
    keywordstyle=\color{blue!90}\bfseries,        %代码关键字的颜色为蓝色,粗体
    backgroundcolor=\color[rgb]{1,1,1},%\color[RGB]{245,245,244},   %代码背景色 \color[rgb]{0.91,0.91,0.91}
    framextopmargin=2pt,framexbottommargin=2pt,abovecaptionskip=-3pt,belowcaptionskip=3pt,
%    xleftmargin=4em,xrightmargin=4em, % 设定listing左右的空白
%    %language={[ISO]C++},       %language为,还有{[Visual]C++}
%    alsolanguage=Java,
%    %alsolanguage=[ANSI]C,      %可以添加很多个alsolanguage,如alsolanguage=matlab,alsolanguage=VHDL等
%    %alsolanguage= tcl,
%    alsolanguage= XML,
%    keepspaces=true,            %
%    breakindent=22pt,           %
%    breakindent=4em,            %
%    showspaces=false,           %
%    flexiblecolumns=true,       %
%    breakautoindent=true,       %
%    aboveskip=1em,              %代码块边框
%    %% added by http://bbs.ctex.org/viewthread.php?tid=53451
%    fontadjust,
%    texcl=true,
%    escapeinside=``,            %在``里显示中文
%    %escapebegin=\begin{CJK*}{GBK}{hei},escapeend=\end{CJK*},
%    % 设定中文冲突,断行,列模式,数学环境输入,listing数字的样式
%    extendedchars=false,columns=flexible,
%    mathescape=false,
%    % numbersep=-1em,
    emph={label}
}

\renewcommand{\ttdefault}{pcr}

\definecolor{darkgreen}{cmyk}{0.7, 0, 1, 0.5}
\definecolor{darkblue}{rgb}{0.1, 0.1, 0.5}
\lstdefinelanguage{diff}
{
    keywords={+, -, \ , @@, diff, index, new},
    sensitive=false,
    morecomment=[l][""]{\ },
    morecomment=[l][\color{darkgreen}]{+},
    morecomment=[l][\color{red}]{-},
    morecomment=[l][\color{darkblue}]{@@},
    morecomment=[l][\color{darkblue}]{diff},
    morecomment=[l][\color{darkblue}]{index},
    morecomment=[l][\color{darkblue}]{new},
    morecomment=[l][\color{darkblue}]{similarity},
    morecomment=[l][\color{darkblue}]{rename},
}

\lstdefinelanguage{JavaScript}{
  keywords={typeof, new, true, false, catch, function, return, null, catch, switch, var, if, in, while, do, else, case, break},
  keywordstyle=\color{blue}\bfseries,
  ndkeywords={class, export, boolean, throw, implements, import, this},
  ndkeywordstyle=\color{darkgray}\bfseries,
  identifierstyle=\color{black},
  sensitive=false,
  comment=[l]{//},
  morecomment=[s]{/*}{*/},
  %commentstyle=\color{purple}\ttfamily,
  commentstyle=\color{green}\ttfamily,
  stringstyle=\color{red}\ttfamily,
  morestring=[b]',
  morestring=[b]"
}

\usepackage{tabularx} % long table
\usepackage{booktabs,longtable} % table in seperate pages.


\ifxetex % xelatex
\else
    %The cmap package is intended to make the PDF files generated by pdflatex "searchable and copyable" in acrobat reader and other compliant PDF viewers.
    \usepackage{cmap}%
\fi
% ============================================
% Check for PDFLaTeX/LaTeX
% ============================================
\newcommand{\outengine}{xetex}
\newif\ifpdf
\ifx\pdfoutput\undefined
  \pdffalse % we are not running PDFLaTeX
  \ifxetex
    \renewcommand{\outengine}{xetex}
  \else
    \renewcommand{\outengine}{dvipdfmx}
  \fi
\else
  \pdfoutput=1 % we are running PDFLaTeX
  \pdftrue
  \usepackage{thumbpdf}
  \renewcommand{\outengine}{pdftex}
  \pdfcompresslevel=9
\fi
\usepackage[\outengine,
    bookmarksnumbered, %dvipdfmx
    %% unicode, %% 不能有unicode选项,否则bookmark会是乱码
    colorlinks=true,
    citecolor=red,
    urlcolor=blue,        % \href{...}{...} external (URL)
    filecolor=red,      % \href{...} local file
    linkcolor=black, % \ref{...} and \pageref{...}
    breaklinks,
    pdftitle={\doctitle},
    pdfauthor={\docauthor},
    pdfsubject={\docsubject},
    pdfkeywords={\dockeywords},
    pdfproducer={Latex with hyperref},
    pdfcreator={pdflatex},
    %%pdfadjustspacing=1,
    pdfborder=1,
    pdfpagemode=UseNone,
    pagebackref,
    bookmarksopen=true]{hyperref}

% --------------------------------------------
% Load graphicx package with pdf if needed 
% --------------------------------------------
\ifxetex    % xelatex
    \usepackage{graphicx}
\else
    \ifpdf
        \usepackage[pdftex]{graphicx}
        \pdfcompresslevel=9
    \else
        \usepackage{graphicx} % \usepackage[dvipdfm]{graphicx}
    \fi
\fi

%% \DeclareGraphicsRule{.jpg}{eps}{.bb}{}
%% \DeclareGraphicsRule{.png}{eps}{.bb}{}
\graphicspath{{./} {figures/}}
\usepackage{flafter} % 防止图形在文字前

%%%% 字体:
%Adobe Heiti Std 和 Adobe Song Std是砖头公司出的两款超pp的字体,有人把它们用在latex排版中,效果超级好。
%windows版在 Program Files/Adobe/Acrobat 8.0/Resource/CIDFont 下。

%sudo mkdir /usr/share/fonts/adobe
%sudo cp DIR2adobefonts/*.otf /usr/share/fonts/adobe
%sudo chmod 644 /usr/share/fonts/adobe/*.otf # 当前用户读写,当前组用户读写,其他用户只读

%cd /usr/share/fonts/adobe/
%sudo mkfontscale #(创建fonts.scale文件,控制字体旋转缩放)
%sudo mkfontdir #(创建fonts.dir文件,控制字体粗斜体产生)
%sudo fc-cache -fv # (建立字体缓存信息,也就是让系统认识认识)
%fc-list :lang=zh-cn # 看看装上没


%安装 LaTeX+XeTeX环境的过程, 你也使用Emacs来编辑TeX文件的话, 那么一定要安上AUCTeX这个扩展
%sudo apt-get install texlive texlive-latex-extra texlive-xetex lmodern # 首先是LaTeX与XeTeX的安装

%sudo apt-get install auctex

%安装好以后, 重点是配置.emacs文件, 因为AUCTeX本身是不支持通过xelatex编译的.
%;; AUCTeX
%(defun auctex ()
  %(add-to-list 'TeX-command-list '("XeLaTeX" "%`xelatex%(mode)%' %t; %`xelatex%(mode)%' %t" TeX-run-TeX nil t)) ;; 这里我编译了两次
    %(setq TeX-command-default "XeLaTeX") ;; 设定默认编译命令为XeLaTeX
    %(setq TeX-save-query nil)            ;; 保存之前不询问
    %(setq TeX-show-compilation t))       ;; 在新窗口显示编译过程
%(add-hook 'LaTeX-mode-hook 'auctex)

%(custom-set-variables
 %'(TeX-output-view-style (quote (("^dvi$nnnnnnn" ("^landscape$" "^pstricks$\\|^pst-\\|^psfrag$") "%(o?)dvips -t landscape %d -o && gv %f") ("^dvi$" "^pstricks$\\|^pst-\\|^psfrag$" "%(o?)dvips %d -o && gv %f") ("^dvi$" ("^a4\\(?:dutch\\|paper\\|wide\\)\\|sem-a4$" "^landscape$") "%(o?)xdvi %dS -paper a4r -s 0 %d") ("^dvi$" "^a4\\(?:dutch\\|paper\\|wide\\)\\|sem-a4$" "%(o?)xdvi %dS -paper a4 %d") ("^dvi$" ("^a5\\(?:comb\\|paper\\)$" "^landscape$") "%(o?)xdvi %dS -paper a5r -s 0 %d") ("^dvi$" "^a5\\(?:comb\\|paper\\)$" "%(o?)xdvi %dS -paper a5 %d") ("^dvi$" "^b5paper$" "%(o?)xdvi %dS -paper b5 %d") ("^dvi$" "^letterpaper$" "%(o?)xdvi %dS -paper us %d") ("^dvi$" "^legalpaper$" "%(o?)xdvi %dS -paper legal %d") ("^dvi$" "^executivepaper$" "%(o?)xdvi %dS -paper 7.25x10.5in %d") ("^dvi$" "." "%(o?)xdvi %dS %d") ("^pdf$" "." "acroread %o %(outpage)") ("^html?$" "." "netscape %o")))))

%最后那个有点长, 主要是没有找到合适的方法像添加XeLaTeX一样只需要写新增的条目, 所以这里就把原有的和修改以后的都写了出来. 其实只改了一个地方, 已经用蓝色标注出来了, 就是在使用C-c C-v预览PDF文件的时候使用什么软件来打开. 我这里就是acroread, 你用的其它的话, 可以相应修改.
%这样修改好以后, 以后就可以直接使用C-c C-c编译, C-c C-v预览, C-c `在错误间跳转了.

%但是TeX Live中的install-info文件会导致源安装AUCTeX的时候失败, 所以如果是先安装的TeX Live, 再安装AUCTeX, 就需要先把TeX Live的install-info"消灭"掉: 
%sudo mv /usr/local/bin/install-info /usr/local/bin/install-info.bak

%-------------------------------------------------------------------
%终于搞定在emacs+auctex中设置xelatex为默认编译命令!

%只要在在~/.emacs中加上

%(add-hook 'LaTeX-mode-hook (lambda()
    %(add-to-list 'TeX-command-list '("XeLaTeX" "%`xelatex%(mode)%' %t" TeX-run-TeX nil t))
    %(setq TeX-command-default "XeLaTeX")
    %(setq TeX-save-query  nil )
    %(setq TeX-show-compilation t)
    %))

%第一行参考auctex的手册auctex.pdf,版本是11.84 ;
%(add-to-list 'TeX-command-list '("XeLaTeX" "%`xelatex%(mode)%' %t" TeX-run-TeX nil t)) 会在Command 这一栏中增加了XeLaTeX这一项命令;
%(setq TeX-command-default "XeLaTeX")  则使得以后用C-c C-c就是默认用xelatex 命令编译tex文档;
%(setq TeX-save-query  nil ) 这一行不用确认保存就开始执行编译;
%(setq TeX-show-compilation t)  这一行是看到编译的滚动信息。
%现在还是在latex-mode下配置,下一步看能否在pdflatex-mode 下配置。


%(add-to-list 'TeX-command-list '("XeLaTeX" "%`xelatex%(mode)%' %t" TeX-run-TeX nil t)) 这一行中的"%`xelatex%(mode)%' %t"
%写成"xelatex  %t" 已经可以了。
